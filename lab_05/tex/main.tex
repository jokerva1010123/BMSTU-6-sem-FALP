
\section*{Практические задания}

\subsection*{Задание 1}

Напишите функцию, которая уменьшает на 10 все числа из списка-аргумента этой функции, проходя по верхнему уровню списковых ячеек. \ (* Список смешанный структурированный).

\begin{lstlisting}
(defun minus-ten (lst) 
(
    mapcar #'(lambda (el) 
        ( cond
                ((numberp el) (- el 10)
                (T el))
        ) lst
    )
)
\end{lstlisting}

\subsection*{Задание 2}

Написать функцию которая получает как аргумент список чисел, а возвращает список квадратов этих чисел в том же порядке.

\begin{lstlisting}
(defun sqr-lst (lst)
(
    mapcar #'(lambda (el)
        (* el el)) lst
    )
)
\end{lstlisting}

\subsection*{Задание 3}

Напишите функцию, которая умножает на заданное число-аргумент все числа из заданного списка-аргумента, когда:
a) все элементы списка --- числа,
б) элементы списка -- любые объекты.

a) 
\begin{lstlisting}
(defun mul-lst (lst num) 
(
    mapcar #'(lambda (el) 
        (* el num)) lst
    )
)
\end{lstlisting}
\newpage
б)
\begin{lstlisting}
(defun mul-lst (lst num) (
    mapcar #'(lambda (el) (
        (cond ((numberp el) (* el num)
            (T el))
        ) lst
    )
)
\end{lstlisting}

\subsection*{Задание 4}

Написать функцию, которая по своему списку-аргументу lst определяет является ли он палиндромом (то есть равны ли lst и (reverse lst)), для одноуровнего смешанного списка.

\begin{lstlisting}
(defun list_without_last (lst res)
	(cond
		((null (cdr lst)) res)
		(t (list_without_last (cdr lst) (cons (car lst) res)))
))

(defun is_palindrome (lst)
	(cond 
		((null (cdr lst)) t)
		((eql (car lst) (car (last lst)))
			(is_palindrome (list_without_last (cdr lst) ())))
))
\end{lstlisting}

\subsection*{Задание 5}

Используя функционалы, написать предикат set-equal, который возвращает t, если два его множества-аргумента (одноуровневые списки) содержат одни и те же элементы, порядок которых не имеет значения.
\begin{lstlisting}
(defun set-equal (set1 set2)
	(cond 
		((null set1) (null set2)) 
		((null set2) Nil) 
		(t (set-equal (cdr set1) (remove (car set1) set2)))
))
\end{lstlisting}

\subsection*{Задание 6}

Напишите функцию, select-between, которая из списка-аргумента, содержащего только числа, выбирает только те, которые расположены между двумя указанными числами  границами-аргументами и возвращает их в виде списка (упорядоченного по возрастанию (+ 2 балла)).

\begin{lstlisting}
(defun select-between (lst left right) (
    cond ((< right left) Nil)
    (T (sort (reduce #'(lambda (result el)
             (cond ((and (> el left) (> right el))
                (cons el result))
            (T result))
        )lst :initial-value '()) #'<
))))
\end{lstlisting}

\subsection*{Задание 7}

Написать функцию, вычисляющую декартово произведение двух своих списковаргументов. ( Напомним, что А х В это множество всевозможных пар (a b), где а принадлежит А, принадлежит В.)

\begin{lstlisting}
(defun decart (lst1 lst2) (
    mapcan #'(lambda (x)
        (mapcar #'(lambda (y)
            (list x y)) lst2)) lst1
))
\end{lstlisting}

\subsection*{Задание 8}

Почему так реализовано reduce, в чем причина?

\begin{lstlisting}
(reduce #'+ ()) -> 0
(reduce #'* ()) -> 1
\end{lstlisting}

Если список пуст, а начальное значение не задано, то вызывается функция без агрументов, в reduce возвращает то, что вернёт функция. Функция сложения без аргументов возвращает 0, а функцуия умножения возвращает 1.


\subsection*{Задание 9}

Пусть list-of-list список, состоящий из списков. Написать функцию, которая
вычисляет сумму длин всех элементов list-of-list (количество атомов), т.е. например
для аргумента
 ((1 2) (3 4)) -> 4.

\begin{lstlisting}
(defun len-list-of-lists (lst) (
    apply #'+ (
        mapcar #'(lambda (el)
            (cond ((listp el) (len-list-of-lists el))
            (T 1))
        ) lst
)))
\end{lstlisting}