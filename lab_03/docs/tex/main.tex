\section*{Теоретические вопросы}
\subsection*{1. Базис Lisp}

Базис -- это минимальный набор инструментов языка и стркутур данных, который позволяет решить любые задачи.

Базис Lisp :

\begin{itemize}
	\item[$-$] атомы и структуры (представляющиеся бинарными узлами);
	\item[$-$] базовые (несколько) функций и функционалов: встроенные — примитивные 
	функции (atom, eq, cons, car, cdr); специальные функции и функционалы (quote, 
	cond, lambda, eval, apply, funcall).
\end{itemize}
	
Функцией называется правило, по которому каждому значению одного или нескольких  аргументов ставится в соответствие конкретное значение результата.

Функционалом, или функцией высшего порядка называется функция, аргументом или  результатом которой является другая функция.

\subsection*{2. Классификация функций}

Функции в языке {\texttt{Lisp}}:
\begin{itemize}
	\item Чистые математические функции (имеют фиксированное количество аргументов, сначала выяисляются все аргументы, а только потом к ним применяется функция);
	\item Рекурсивные функции (основной способ выполнения повторных вычислений);
	\item Специальные функции, или формы (могут принимать произвольное количество аргументов, или аргументы могут обрабатываться по-разному);
	\item Псевдофункции (создают «эффект», например, вывод на экран);
	\item Функции с вариантами значений, из которых выбирается одно;
	\item Функции высших порядков, или функционалы --  функции, аргументом или  результатом которых является другая функция (используются для построения синтаксически управляемых программ)
	\item Базисные функции --- минимальный набор функций, позволяющих решить любую задачу.
\end{itemize}

Также базисные и функции ядра можно классифицировать с точки \\зрения действий.
\begin{enumerate}
	\item Селекторы --- переходят по соответствующему указателю списковой ячейки.
	\item Конструкторы --- создают структуры данных.
	\item Предикаты --- позволяют классифицировать или сравнивать структуры.
\end{enumerate}

\subsection*{3. Способы создание функций}

\begin{itemize}
	\item С помощью lambda. После ключевого слова указывается лямбда-список и тело функции. 
	\begin{lstlisting}
(lambda (x y) (+ x y))
	\end{lstlisting}
	Для применения используются лямбда-выражения.
	\begin{lstlisting}
((lambda (x y) (+ x y)) 1 2)
	\end{lstlisting}
	\item С помощью defun. Используется для неоднократного применения функции (в том числе рекурсивного вызова).
	\begin{lstlisting}
(defun sum (x y) (+ x y))
(sum 1 2)
	\end{lstlisting}

\end{itemize}

\subsection*{4. Работа функций Cond, if, and/or}
\begin {itemize}
\item Функция \texttt{cond}

\hspace{1cm} Синтаксис:
\begin{lstlisting}
(cond 
	(condition-1 expression-1)
	(condition-2 expression-2)
	...
	(condition-n expression-n))
\end{lstlisting}

\hspace{1cm}По порядку вычисляются и проверяются на равенство с \texttt{Nil} предикаты. Для первого предиката, который не равен \texttt{Nil}, вычисляется находящееся с ним в списке выражение и возвращается его значение. Если все предикаты вернут \texttt{Nil}, то и \texttt{cond} вернет \texttt{Nil}.

\hspace{1cm} Примеры:
\begin{lstlisting}
(cond (Nil 1) (2 3)) -> 3
(cond (Nil 1) (Nil 2)) -> %\texttt{Nil}%
\end{lstlisting}

\item Функция \texttt{if}

\hspace{1cm} Синтаксис: (if condition t-expression f-expression)

\hspace{1cm} Если вычисленный предикат не \texttt{Nil}, то выполняется \texttt{t-expression}, иначе - \texttt{f-expression}.

\hspace{1cm} Примеры:

\begin{lstlisting}
(if Nil 2 3) -> 3
(if 0 2 3) -> 2
\end{lstlisting}

\item Функция \texttt{and}

\hspace{1cm} Синтаксис: (and expression-1 expression-2 ... expression-n)

\hspace{1cm}Функция возвращает первое \texttt{expression}, результат вычисления которого = \texttt{Nil}. Если все не \texttt{Nil}, то возвращается результат вычисления последнего выражения.

\hspace{1cm}Примеры:
\begin{lstlisting}
(and 1 Nil 2) -> %\texttt{Nil}%
(and 1 2 3) -> 3
\end{lstlisting}

\item Функция \texttt{or}

\hspace{1cm}Синтаксис: (or expression-1 expression-2 ... expression-n)

\hspace{1cm}Функция возвращает первое \texttt{expression}, результат вычисления которого не \texttt{Nil}. Если все \texttt{Nil}, то возвращается \texttt{Nil}.

\hspace{1cm}Примеры:
\begin{lstlisting}
(or Nil Nil 2) -> 2
(or 1 2 3) -> 1
\end{lstlisting}
\end{itemize}
\newpage
\section*{Практические задания}

\subsection*{1. Задание 1}

Написать функцию, которая принимает целое число и возвращает первое четное число, не меньшее аргумента.
\begin{lstlisting}
(defun f1 (x) (if (oddp x) (+ x 1) x))
(f1 0) => 0
(f1 1) => 2
(f1 -3) => -2
\end{lstlisting}

\subsection*{2. Задание 2}

Написать функцию, которая принимает число и возвращает число того же знака, но с модулем на 1 больше модуля аргумента.

\begin{lstlisting}
(defun f2 (x) (+ x (if (< x 0) -1 1)))
(f2 -5) => -6
(f2 0.0) => 1.0
(f2 2/3) => 5/3
\end{lstlisting}

\subsection*{3. Задание 3}

Написать функцию, которая принимает два числа и возвращает список из этих чисел, расположенный по возрастанию.

\begin{lstlisting}
(defun f3 (x1 x2) (if (> x1 x2) (list x2 x1) (list x1 x2)))
(f3 -1 2) => (-1 2)
(f3 3 1) => (1 3)
(f3 2 2.0) => (2 2.0)
\end{lstlisting}

\subsection*{4. Задание 4}

Написать функцию, которая принимает три числа и возвращает Т только тогда, когда первое число расположено между вторым и третьим.
\begin{lstlisting}
(defun f4 (x1 x2 x3) (and (< x2 x1) (< x1 x3)))
(f4 2 1 3) => T
(f4 1 2 3) => NIL
(f4 3 1 2) => NIL
\end{lstlisting}

\subsection*{5. Задание 5}

Каков результат вычисления следующих выражений?

\begin{lstlisting}
(and 'fee 'fie 'foe) => FOE
(or 'fee 'fie 'foe) => FEE
(or nil 'fie 'foe) => FIE
(and nil 'fie 'foe) => NIL
(and (equal 'abc 'abc) 'yes) => YES
(or (equal 'abc 'abc) 'yes) => T
\end{lstlisting}

\subsection*{6. Задание 6}

Написать предикат, который принимает два числа-аргумента и возвращает Т, если первое число не меньше второго.

\begin{lstlisting}
(defun gep (x1 x2) (>= x1 x2))
(gep 2 2.0) => T
(gep -1 2/3) => NIL
\end{lstlisting}

\subsection*{7. Задание 7}

Какой из следующих двух вариантов предиката ошибочен и почему?

\begin{lstlisting}
	(defun pred1 (x) (and (numberp x) (plusp x))) 
	(defun pred2 (x) (and (plusp x) (numberp x)))
\end{lstlisting}

Предикат numberp вырабатывает T, если значение его аргумента – числовой атом, и NIL в противном случае. plusp проверяет, является ли одиночное вещественное число большим чем ноль.

Таким образом, идея приведенных вариантов предиката заключается в том, чтобы проверить, является ли переданный ему аргумент числом, большим нуля.

При вычислении функционального обращения (and e1 e2 … en) последовательно слева направо вычисляются аргументы функции ei – до тех пор, пока не  встретится значение, равное NIL. В этом случае вычисление прерывается и значение функции равно NIL. Если же были вычислены все значения ei и  оказалось, что все они отличны от NIL, то результирующим значением функции and будет значение последнего выражения en.

Таким образом, корректным является первый вариант предиката. Первым будет вычислено значение аргумента e1 = (numberp x), которое проверит, является ли переданный аргумент числовым атомом. Если это не так, то e1 = (numberp x) вернет Nil, на чем вычисление функции and прервется, и результатом всего предиката pred1 будет NIL. Если же переданный аргумент является числовым атомом, то следующим будет вычислено значение аргумента e2 = (plusp x). Это выражение проверит, является ли переданный числовой атом (здесь мы уже уверены, что переданный аргумент -- числовой атом) больше нуля, и станет результатом всего предиката pred1.

Второй вариант же является ошибочным. В нем первым будет вычислено значение аргумента e1 = (plusp x), но plusp принимает только числовой атом, и если x не является числовым атомом, то вычисление всего предиката pred2 завершится с ошибкой "неверный тип".

\subsection*{8. Задание 8}

Решить задачу 4, используя для ее решения конструкции: только IF, только COND, только AND/OR.

(and x y) можно представить как (cond (x y)); (or x y) можно представить как (cond (x) (y)).

Все приведенные ниже функции на тестах из предыдущего номера выдают те же результаты.

\begin{lstlisting}
	; cond
	(defun f4_1 (x1  x2 x3) (cond ((< x2 x1) (< x1 x3))))
	; if
	(defun f4_2 (x1  x2 x3) (if (< x2 x1) (< x1 x3)))
	; or 
	(defun f4_3 (x1 x2 x3) (not (or (>= x2 x1) (>= x1 x3))))
\end{lstlisting}

\subsection*{9. Задание 9}

Переписать функцию how-alike, приведенную в лекции и использующую COND, используя только конструкции IF, AND/OR.

Исходная функция:
\begin{lstlisting}
(defun how-alike (x y) 
	(cond 
		((or (= x y) (equal x y)) 'the_same) 
		((and (oddp x) (oddp y)) 'both_odd) 
		((and (evenp x) (evenp y)) 'both_even) 
		(t 'diff)
	)
)
(how-alike 1 1) => THE_SAME
(how-alike 2.5 2.5) => THE_SAME
(how-alike -2/3 -4/6) => THE_SAME
(how-alike 3 5) => BOTH_ODD
(how-alike -4 6) => BOTH_EVEN
(how-alike 1 2) => DIFF
\end{lstlisting}

(= работает только с числами, причем они могут быть различных типов; oddp и evenp работают только с целыми числами)

Переписанная функция:
\begin{lstlisting}
; if, and, or
(defun how-alike (x y) 
	(if (or (= x y) (equal x y)) 'the_same 
		(if (and (oddp x) (oddp y)) 'both_odd 
			(if (and (evenp x) (evenp y)) 'both_even 
				'diff
			)
		)
	)
)
; and, or
(defun how-alike (x y) 
	(or 
		(and 
			(or (= x y) (equal x y)) 'the_same
		) 
		(and 
			(and (oddp x) (oddp y)) 'both_odd
		) 
		(and 
			(and (evenp x) (evenp y)) 'both_even
		) 
		'diff
	))
	
; if		
(defun how-alike (x y) 
	(if 
		(if (= x y) T (equal x y))
		'the_same
		(if 
			(if (oddp x) (oddp y))
			'both_odd 
			(if 
				(if (evenp x) (evenp y))
				'both_even
				'diff
			)
		)
	)
)
\end{lstlisting}