\section*{1. запустить среду \texttt{Visual Prolog 5.2}.  Настроить утилиту \texttt{TestGoal}. Запустить тестовую программу, проанализировать реакцию системы и множество ответов. Разработать свою программу – <<Телефонный справочник>>. Протестировать работы программы.}

\inputminted[
frame=single,
framesep=2mm,
baselinestretch=1.2,
bgcolor=white,
fontsize=\footnotesize,
linenos
]{prolog}{../src/main.pro}

Программа на языке \texttt{Prolog} представляет собой базу знаний и вопрос. База знаний --- набор фактов и правил, которые формируют базу знаний о предметной области. Факт --- частный случай правила, состоит только из заголовка и с его помощью фиксируется \textit{истиностное} отношение между объектами предметной области. С помощью правила также фиксируются знания, однако правила обладают телом, в котором фиксируется условие истинности правила. При поиске ответа на вопрос \texttt{Prolog} рассматривает альтернативные варианты и находит все возможные решения --- множества значений переменных, при которых на поставленный вопрос можно ответить \texttt{''да''}.

Программа состоит из разделов (структура программы), каждый имеет свой заголовок:
\begin{itemize}
	\item \texttt{constants} --- раздел описания констант.
	\item \texttt{domains} --- раздел описания доменов.
	\item \texttt{database} --- раздел описания предикатов внутренней базы данных.
	\item \texttt{predicates} --- раздел описания предикатов.
	\item \texttt{clauses} --- раздел описания предложений базы знаний.
	\item \texttt{goal} --- раздел описания внутренней цели (вопроса).
\end{itemize}

В программе не обязательно должны быть описаны все разделы.