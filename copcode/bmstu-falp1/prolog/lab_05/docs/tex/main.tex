\textbf{Задание:} в одной программе написать правила, позволяющие найти:
\begin{enumerate}
	\item Максимум из двух чисел:
	\begin{itemize}
		\item Без использования отсечения;
		\item С использованием отсечения;
	\end{itemize}
	\item Максимум из трех чисел:
	\begin{itemize}
		\item Без использования отсечения;
		\item С использованием отсечения.
	\end{itemize}
\end{enumerate}

Убедиться в правильности результатов. Для каждого случая из пункта 2 обосновать необходимость всех условий тела. Для одного из вариантов ВОПРОСА и каждого варианта задания 2 составить таблицу, отражающую конкретный порядок работы системы.

Так как резольвента хранится в виде стека, то состояние резольвенты требуется отображать в столбик: вершина – сверху! Новый шаг надо начинать с нового состояния резольвенты!

Требуется ответить на вопрос: <<За счет чего может быть достигнута эффективность работы системы?>>
\clearpage

\inputminted[
frame=single,
framesep=2mm,
baselinestretch=1.2,
bgcolor=white,
fontsize=\footnotesize,
linenos,
breaklines
]{prolog}{../src/main.pro}

\clearpage

\section*{Теоретические вопросы}

\subsection*{1. Какое первое состояние резольвенты?}

Стек, который содержит конъюнкцию целей, истинность которых система должна доказать, называется резольвентой. Первое состояние резольвенты -- вопрос.

\subsection*{2. В каком случае система запускает алгоритм унификации? (Как эту необходимость на формальном уровне распознает система?)}

Унификация – необходима для того, чтобы определить дальнейший путь поиска решений. Унификация заканчивается конкретизацией части переменных.

\subsection*{3-4. Каковы назначение и результат использования алгоритма унификации?}

Алгоритм унификации -- основной шаг с помощью которого система отвечает на вопросы унификации. На вход алгоритм принимает два терма, возвращает флаг успешности унификации, и если успешно, то подстановку.

\subsection*{5. В каких пределах программы уникальны переменные?}

Областью действия переменной в Прологе является одно предложение. В разных предложениях может использоваться одно имя переменной для обозначения разных объектов.

\subsection*{6. Как применяется подстановка, полученная с помощью алгоритма унификации?}

Пусть дан терм: $(X_1, X_2,\dots, X_n)$, подстановка -- множество пар, вида: ${X_i = t_i}$, где $X_i$ -- переменная, а $t_i$ -- терм. В ходе выполнения программы выполняется связывание переменных с различными объектами, этот процесс называется конкретизацией. Это относится только к именованным переменным. Анонимные переменные не могут быть связаны со значением.

\subsection*{7. Как изменяется резольвента?}

Резольвента меняется в два этапа:

\begin{enumerate}
	\item редукция -- замена подцели телом того правила, с заголовком которого успешно унифицируется данная подцель;
	
	\item применение ко всей резольвенте подстановки.
\end{enumerate}

Резольвента уменьшается, если удаётся унифицировать подцель с фактом. Система отвечает <<Да>>, только когда резольвента становится пустой.

\subsection*{8. В каких случаях запускается механизм отката?}

Механизм отката, который осуществляет откат программы к той точке, в которой выбирался унифицирующийся с последней подцелью дизъюнкт. Для этого точка, где выбирался один из возможных унифицируемых с подцелью дизъюнктов, запоминается в специальном стеке, для последующего возврата к ней и выбора альтернативы в случае неудачи. При откате все переменные, которые были означены в результате унификации после этой точки, опять становятся свободными.

