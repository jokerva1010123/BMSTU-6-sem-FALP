\section*{Задание 1}

Написать функцию, которая принимает целое число и возвращает первое четное число, не меньшее аргумента.

\begin{lstinputlisting}[
	caption={Задание 1},
	label={lst:t1},
	style={lsp},
	linerange={1-4},
	]{../src/main.lsp}
\end{lstinputlisting}

\section*{Задание 2}

Написать функцию, которая принимает число и возвращает число того же знака, но с модулем на 1 больше модуля аргумента.

\begin{lstinputlisting}[
	caption={Задание 2},
	label={lst:t2},
	style={lsp},
	linerange={6-9},
	]{../src/main.lsp}
\end{lstinputlisting}

\section*{Задание 3}

Написать функцию, которая принимает два числа и возвращает список из этих чисел, расположенный по возрастанию.

\begin{lstinputlisting}[
	caption={Задание 3},
	label={lst:t3},
	style={lsp},
	linerange={11-12},
	]{../src/main.lsp}
\end{lstinputlisting}

\section*{Задание 4}

Написать функцию, которая принимает три числа и возвращает \texttt{Т} только тогда, когда первое число расположено между вторым и третьим.

\begin{lstinputlisting}[
	caption={Задание 4},
	label={lst:t4},
	style={lsp},
	linerange={14-17},
	]{../src/main.lsp}
\end{lstinputlisting}

\section*{Задание 5}

Каков результат вычисления следующих выражений?

\begin{lstinputlisting}[
	caption={Задание 5},
	label={lst:t5},
	style={lsp},
	linerange={20-25},
	]{../src/main.lsp}
\end{lstinputlisting}

\section*{Задание 6}

Написать предикат, который принимает два числа аргумента и возвращает \texttt{Т}, если первое число не меньше второго.

\begin{lstinputlisting}[
	caption={Задание 6},
	label={lst:t6},
	style={lsp},
	linerange={28-31},
	]{../src/main.lsp}
\end{lstinputlisting}

\section*{Задание 7}

Какой из следующих двух вариантов предиката
ошибочен и почему?

\begin{lstinputlisting}[
	caption={mystery},
	label={lst:func},
	style={lsp},
	linerange={33-37},
	]{../src/main.lsp}
\end{lstinputlisting}

Второй предикат ошибочен, потому что в случае, если на вход функции будет подан аргумент, не являющийся числом, к нему будет применена
функция {\texttt{plusp}}, определяющая, является ли число положительным. Она
определена только для чисел, поэтому в случае нечислового аргумента интерпретатор выдаст ошибку.

\section*{Задание 8}

Решить задачу 4, используя для ее решения конструкции {\texttt{IF, COND, AND/OR}}.

\begin{lstinputlisting}[
	caption={Задание 8},
	label={lst:t8},
	style={lsp},
	linerange={39-53},
	]{../src/main.lsp}
\end{lstinputlisting}

\section*{Задание 9}

Переписать функцию {\texttt{how-alike}}, приведенную
в лекции и использующую {\texttt{COND}}, используя конструкции {\texttt{IF, AND/OR}}.

\begin{lstinputlisting}[
	caption={Задание 9},
	label={lst:t9},
	style={lsp},
	linerange={55-81},
	]{../src/main.lsp}
\end{lstinputlisting}

\clearpage
\section*{Контрольные вопросы}

\subsection*{1. Базис языка {\texttt{Lisp}}}

Базис языка представлен:
\begin{itemize}
	\item структурами, атомами;
	\item функциями:\\
	{\texttt{atom, eq, cons, car, cdr,}}\\
	{\texttt{cond, quote, lambda, eval, label}}.
\end{itemize}

\subsection*{2. Классификация функций языка {\texttt{Lisp}}}

Функции в языке {\texttt{Lisp}}:
\begin{itemize}
	\item чистые (с фиксированным количеством аргументов) -- математические функции;
	\item рекурсивные функции;
	\item специальные функции -- формы (принимают произвольное количество аргументов или по разному обрабатывают аргументы);
	\item псевдофункции (создающие <<эффект>> - отображающие на экране процесс обработки данных и т. п.);
	\item функции с вариативными значениями, выбирающие одно значение;
	\item функции высших порядков -- функционалы (используются для построения синтаксически управляемых программ).
\end{itemize}

По назначению функции разделяются следующим образом:
\begin{enumerate}
	\item конструкторы --- создают значение (\texttt{cons}, например);
	\item селекторы --- получают доступ по адресу (\texttt{car}, \texttt{cdr});
	\item предикаты --- возвращают \texttt{Nil}, \texttt{T}.
\end{enumerate}

\subsection*{3. Способы создания функций}

С помощью макро определения \texttt{defun} или с использованием Лямбда-нотации (функция без имени).

\subsection*{Работа функций \texttt{and}, \texttt{or}, \texttt{if}, \texttt{cond}}

\subsubsection*{Функция \texttt{and}}

Синтаксис: \code{(and expression-1 expression-2 ... expression-n)}

Функция возвращает первое \texttt{expression}, результат вычисления которого = \texttt{Nil}. Если все не \texttt{Nil}, то возвращается результат вычисления последнего выражения.

Примеры:
\begin{lstlisting}[style={scheme}]
(and 1 Nil 2) -> %\texttt{Nil}%
(and 1 2 3) -> 3
\end{lstlisting}

\subsubsection*{Функция \texttt{or}}

Синтаксис: \code{(or expression-1 expression-2 ... expression-n)}

Функция возвращает первое \texttt{expression}, результат вычисления которого не \texttt{Nil}. Если все \texttt{Nil}, то возвращается \texttt{Nil}.

Примеры:
\begin{lstlisting}[style={scheme}]
(or Nil Nil 2) -> 2
(or 1 2 3) -> 1
\end{lstlisting}


\subsubsection*{Функция \texttt{if}}

Синтаксис: \code{(if condition t-expression f-expression)}

Если вычисленный предикат не \texttt{Nil}, то выполняется \texttt{t-expression}, иначе - \texttt{f-expression}.

Примеры:

\begin{lstlisting}[style={scheme}]
(if Nil 2 3) -> 3
(if 0 2 3) -> 2
\end{lstlisting}

\subsubsection*{Функция \texttt{cond}}

Синтаксис:
\begin{lstlisting}[style={scheme}]
(cond
  (condition-1 expression-1)
  (condition-2 expression-2)
  ...
  (condition-n expression-n))
\end{lstlisting}

По порядку вычисляются и проверяются на равенство с \texttt{Nil} предикаты. Для первого предиката, который не равен \texttt{Nil}, вычисляется находящееся с ним в списке выражение и возвращается его значение. Если все предикаты вернут \texttt{Nil}, то и \texttt{cond} вернет \texttt{Nil}.

Примеры:
\begin{lstlisting}[style={scheme}]
(cond (Nil 1) (2 3)) -> 3
(cond (Nil 1) (Nil 2)) -> %\texttt{Nil}%
\end{lstlisting}

