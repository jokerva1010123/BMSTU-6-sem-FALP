\textbf{Задание:} Создать базу знаний: <<ПРЕДКИ>>, позволяющую наиболее эффективным способом (за меньшее количество шагов, что обеспечивается меньшим количеством предложений БЗ – правил), и используя разные варианты (примеры) одного вопроса, определить (указать: какой вопрос для какого варианта):

\begin{enumerate}
	\item По имени субъекта определить всех его бабушек (предки 2-го колена);
	\item По имени субъекта определить всех его дедушек (предки 2-го колена);
	\item По имени субъекта определить всех его бабушек и дедушек (предки 2-го колена);
	\item По имени субъекта определить его бабушку по материнской линии (предки 2-го колена);
	\item По имени субъекта определить его бабушку и дедушку по материнской линии (предки 2-го колена).
\end{enumerate}

Минимизировать количество правил и количество вариантов вопросов. Использовать конъюнктивные правила и простой вопрос.

Для одного из вариантов ВОПРОСА и конкретной БЗ составить таблицу, отражающую конкретный порядок работы системы, с объяснениями:

\begin{itemize}
	\item очередная проблема на каждом шаге и метод ее решения,
	\item каково новое текущее состояние резольвенты, как получено,
	\item какие дальнейшие действия? (запускается ли алгоритм унификации? Каких термов? Почему этих?),
	\item вывод по результатам очередного шага и дальнейшие действия.
\end{itemize}

Так как резольвента хранится в виде стека, то состояние резольвенты требуется отображать в столбик: вершина – сверху! Новый шаг надо начинать с нового состояния резольвенты!

\inputminted[
frame=single,
framesep=2mm,
baselinestretch=1.2,
bgcolor=white,
fontsize=\footnotesize,
linenos,
breaklines
]{prolog}{../src/main.pro}

\section*{Теоретические вопросы}

\subsection*{1. В каком случае система запускает алгоритм унификации? (Как эту необходимость на формальном уровне распознает система?)}

Унификация – необходима для того, чтобы определить дальнейший путь поиска решений. Унификация заканчивается конкретизацией части переменных.

\subsection*{2. Каковы назначение и результат использования алгоритма унификации?}

Алгоритм унификации -- основной шаг с помощью которого система отвечает на вопросы унификации. На вход алгоритм принимает два терма, возвращает флаг успешности унификации, и если успешно, то подстановку.

\subsection*{3. Какое первое состояние резольвенты?}

Стек, который содержит конъюнкцию целей, истинность которых система должна доказать, называется резольвентой. Первое состояние резольвенты -- вопрос.

\subsection*{4. Как меняется резольвента?}

Резольвента меняется в два этапа:

\begin{enumerate}
	\item редукция -- замена подцели телом того правила, с заголовком которого успешно унифицируется данная подцель;
	
	\item применение ко всей резольвенте подстановки.
\end{enumerate}

Резольвента уменьшается, если удаётся унифицировать подцель с фактом. Система отвечает <<Да>>, только когда резольвента становится пустой.

\subsection*{5. В каких пределах программы уникальны переменные?}

Областью действия переменной в Прологе является одно предложение. В разных предложениях может использоваться одно имя переменной для обозначения разных объектов.

\subsection*{6. Как применяется подстановка, полученная с помощью алгоритма унификации?}

Пусть дан терм: $(X_1, X_2,\dots, X_n)$, подстановка -- множество пар, вида: ${X_i = t_i}$, где $X_i$ -- переменная, а $t_i$ -- терм. В ходе выполнения программы выполняется связывание переменных с различными объектами, этот процесс называется конкретизацией. Это относится только к именованным переменным. Анонимные переменные не могут быть связаны со значением.

\subsection*{7. В каких случаях запускается механизм отката?}

Механизм отката, который осуществляет откат программы к той точке, в которой выбирался унифицирующийся с последней подцелью дизъюнкт. Для этого точка, где выбирался один из возможных унифицируемых с подцелью дизъюнктов, запоминается в специальном стеке, для последующего возврата к ней и выбора альтернативы в случае неудачи. При откате все переменные, которые были означены в результате унификации после этой точки, опять становятся свободными.
