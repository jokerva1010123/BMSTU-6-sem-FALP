\chapter{Практические задания}

\section{Задание №1}

Решение приложено к отчету на отдельном листе.

\section{Задание №2}

Используя функции $\text{CAR}$ и $\text{CDR}$, написать выражения,
возвращающие:
\begin{enumerate}
    \item второй элемент;

    \vspace{4mm}
    \hfill
    \begin{minipage}{0.81\linewidth}
    \begin{lstlisting}
(car (cdr '(a b c d)))
    \end{lstlisting}
    \end{minipage}

    \item третий элемент;

    \vspace{4mm}
    \hfill
    \begin{minipage}{0.81\linewidth}
    \begin{lstlisting}
(car (cdr (cdr '(a b c d))))
    \end{lstlisting}
    \end{minipage}

    \item четвертый элемент.

    \vspace{4mm}
    \hfill
    \begin{minipage}{0.81\linewidth}
    \begin{lstlisting}
(car (cdr (cdr (cdr '(a b c d)))))
    \end{lstlisting}
    \end{minipage}
\end{enumerate}

\section{Задание №3}

Что будет в результате вычисления выражений?

\begin{enumerate}
    \item Выражение:

        \vspace{4mm}
        \hfill
        \begin{minipage}{0.81\linewidth}
        \begin{lstlisting}
(caadr '((blue cube) (red pyramid)))
        \end{lstlisting}
        \end{minipage}

    ~~~~Результат: red

    \item Выражение:

        \vspace{4mm}
        \hfill
        \begin{minipage}{0.81\linewidth}
        \begin{lstlisting}
(cdar '((abc) (def) (ghi)))
        \end{lstlisting}
        \end{minipage}

    ~~~~Результат: Nil

    \item Выражение:

        \vspace{4mm}
        \hfill
        \begin{minipage}{0.81\linewidth}
        \begin{lstlisting}
(cadr '((abc) (def) (ghi)))
        \end{lstlisting}
        \end{minipage}

    ~~~~Результат: (def)

    \item Выражение:

        \vspace{4mm}
        \hfill
        \begin{minipage}{0.81\linewidth}
        \begin{lstlisting}
(caddr '((abc) (def) (ghi)))
        \end{lstlisting}
        \end{minipage}

    ~~~~Результат: (ghi)
\end{enumerate}

\section{Задание №4}

Напишите результат вычисления выражений:

\vspace{4mm}
\begin{minipage}{0.92\linewidth}
\begin{lstlisting}
;; Выражение                     Результат
(list 'Fred 'and 'Wilma)        ; (Fred and Wilma)
(list 'Fred '(and Wilma))       ; (Fred (and Wilma))
(cons Nil Nil)                  ; (Nil)
(cons T Nil)                    ; (T)
(cons Nil T)                    ; (Nil . T)
(list Nil)                      ; (Nil)
(cons '(T) Nil)                 ; ((T))
(list '(one two) '(free temp))  ; ((one two) (free temp))
(cons 'Fred '(and Wilma))       ; (Fred and Wilma)
(cons 'Fred '(Wilma))           ; (Fred Wilma)
(list Nil Nil)                  ; (Nil Nil)
(list T Nil)                    ; (T Nil)
(list Nil T)                    ; (Nil T)
(cons T (list Nil))             ; (T Nil)
(list '(T) Nil)                 ; ((T) Nil)
(cons '(one two) '(free temp))  ; ((one two) free temp)
\end{lstlisting}
\end{minipage}


\section{Задание №5}

Решение приложено к отчету на отдельном листе.

\chapter{Теоретические вопросы}

\section{Элементы языка: определение, синтаксис, представление в памяти}

Вся информация (данные и программы) в Lisp представляется в виде символьных
выражений --- S-выражений.

\vspace{4mm}
\begin{minipage}{0.92\linewidth}
\begin{lstlisting}
S-выражние ::= <атом>|<точечная пара>
\end{lstlisting}
\end{minipage}

Атомы:
\begin{itemize}
    \item символы --- синтаксически --- набор литер (букв и цифр), начинающийся
        с буквы;
    \item специальные символы \{T, Nil\} обозначают логические константы;
    \item самоопределимые атомы --- натуральные, дробные, вещественные
        числа и строки.
\end{itemize}

Более сложные данные --- точечные пары и списки (структуры).

\vspace{4mm}
\begin{minipage}{0.92\linewidth}
\begin{lstlisting}
Точечные пары ::= (<атом>, <атом>) 
                | (<атом>, <точечная пара>)
                | (<точечная пара>, <атом>)
                | (<точечная пара>, <точечная пара>)
\end{lstlisting}
\end{minipage}

\vspace{4mm}
\begin{minipage}{0.92\linewidth}
\begin{lstlisting}
Список ::= <пустой список> | <непустой список>
<пустой список> ::= () | Nil
<непустой список> ::= (<первый элемент>, <хвост>)
<первый элемент> ::= <S-выражение>
<хвост> ::= <список>
\end{lstlisting}
\end{minipage}

Синтаксически любая структура заключается в круглые скобки:
\begin{itemize}
    \item (A . B) -- точечная пара;
    \item (A) -- список из одного элемента;
    \item Nil или () -- пустой список;
    \item (A . (B . (C . (D ()))))) или (A B C D) -- непустой список;
    \item элементы списка могу являться списками: ((A)(B)(CD)).
\end{itemize}

Любая непустая структура Lisp в памяти представляется списковой ячейкой,
хранящей два указателя: на голову (первый элемент) и хвост (все остальное).

\section{Особенности языка Lisp. Структура программы. Символ апостроф}

Особенности языка Lisp:
\begin{itemize}
    \item символьная обработка данных;
    \item любая программа может интерпретироваться как функция с
        одним или несколькими аргументами;
    \item автоматизированное динамическое распределение памяти, которая
        выделяется блоками;
    \item бестиповый язык;
    \item программа может быть представлена как данные, то есть программа
        может изменять саму себя.
\end{itemize}

Символ апостроф --- сокращеное обозначение функции quote, блокирующей
вычисление своего аргумента.

\section{Базис языка Lisp. Ядро языка}

Базис языка --- минимальный набор конструкций и структур данных, с помощью
которого можно написать любую программу.

Базис Lisp образуют:
\begin{itemize}
    \item атомы;
    \item структуры;
    \item базовые функции;
    \item базовые функционалы.
\end{itemize}

