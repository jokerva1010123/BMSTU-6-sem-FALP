\section*{Задание 1}

Каковы результаты следующих выражений?

\begin{lstinputlisting}[
	caption={Задание 1},
	label={lst:t1},
	style={lsp},
	linerange={2-6},
	]{../src/main.lsp}
\end{lstinputlisting}

\section*{Задание 2}

Каковы результаты следующих выражений?

\begin{lstinputlisting}[
	caption={Задание 2},
	label={lst:t2},
	style={lsp},
	linerange={10-14},
	]{../src/main.lsp}
\end{lstinputlisting}

\section*{Задание 3}

Написать, по крайней мере, два варианта функции, которая возвращает последний элемент своего списка-аргумента.

\begin{lstinputlisting}[
	caption={Задание 3},
	label={lst:t3},
	style={lsp},
	linerange={17-30},
	]{../src/main.lsp}
\end{lstinputlisting}

\section*{Задание 4}

Написать, по крайней мере, два варианта функции, которая возвращает свой список-аргумент без последнего элемента

\begin{lstinputlisting}[
	caption={Задание 4},
	label={lst:t4},
	style={lsp},
	linerange={32-41},
	]{../src/main.lsp}
\end{lstinputlisting}

\section*{Задание 5}

в котором бросаются две правильные кости. Если сумма выпавших очков равна 7 или 11 --- выигрыш, если выпало $(1, 1)$ или $(6, 6)$ --- игрок получает право снова бросить кости, во всех остальных случаях ход переходит ко второму игроку, но запоминается сумма выпавших очков. Если второй игрок не выигрывает абсолютно, то выигрывает тот игрок, у которого больше очков. Результат игры и значения выпавших костей выводить на экран с помощью функции \texttt{print}.

\clearpage
\begin{lstinputlisting}[
	caption={Задание 5},
	label={lst:t5},
	style={lsp},
	linerange={1-44},
	]{../src/dices.lsp}
\end{lstinputlisting}

\section*{Контрольные вопросы}

\subsection*{1. Синтаксическая форма и хранение программы в памяти}

В \texttt{LISP} формы представления программы и обрабатываемых ею данных одинаковы и представляются в виде \texttt{S-выражений}. Поэтому программы могут обрабатывать и преобразовывать другие программы и даже самих себя. В процессе трансляции можно введенное и сформированное в результате вычислений выражение данных проинтерпретировать в качестве программы и непосредственно выполнить. Так как программа представляет собой S-выражение, в памяти она представлена либо как атом (5 указателей; форма представления атома в памяти), либо списковой ячейкой (бинарный узел; 2 указателя).

\subsection*{2. Трактовка элементов списка}

Первый аргумент списка, который поступает на вход интерпретатору, трактуется как имя функции, остальные --- как аргументы этой функции.

\subsection*{3. Порядок реализации программы}

Программа в языке \texttt{LISP} представляется \texttt{S-выражением}, которое передается интерпретатору --- функции \texttt{eval}, которая выводит последний, полученный после обработки S-выражения, результат.
Работа функции \texttt{eval} представлена на картинке ниже.

\subsection*{4. Способы определения функций}

С помощью макро определения \texttt{defun} или с использованием Лямбда-нотации (функция без имени).