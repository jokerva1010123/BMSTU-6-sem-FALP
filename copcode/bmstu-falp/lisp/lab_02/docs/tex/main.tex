\section*{Задание 1}

Составить диаграмму вычисления следующих выражений:

\begin{lstlisting}[style={scheme}]
1. (equal 3 (abs -3))
     3 %\as% 3
     (abs -3)
       -3 %\as% -3
       abs %\at% -3
       3
     equal %\at% 3 и 3
     T
\end{lstlisting}

\begin{lstlisting}[style={scheme}]
2. (equal (+ 1 2) 3)
     (+ 1 2)
       1 %\as% 1
       2 %\as% 2
       + %\at% 1 и 2
       3
     3 %\as% 3
     equal %\at% 3 и 3
     T
\end{lstlisting}

\clearpage
\begin{lstlisting}[style={scheme}]
3. (equal (* 4 7) 21)
     (* 4 7)
       4 %\as% 4
       7 %\as% 7
       * %\at% 4 и 7
       28
     21 %\as% 21
     equal %\at% 28 и 21
     Nil
\end{lstlisting}

\begin{lstlisting}[style={scheme}]
4. (equal (* 2 3) (+ 7 2))
     (* 2 3)
       2 %\as% 2
       3 %\as% 3
       * %\at% 2 и 3
       6
     (+ 7 2)
       7 %\as% 7
       2 %\as% 2
       + %\at% 7 и 2
       9
     equal %\at% 6 и 9
     Nil
\end{lstlisting}

\clearpage
\begin{lstlisting}[style={scheme}]
5. (equal (- 7 3) (* 3 2))
     (- 7 3)
       7 %\as% 7
       3 %\as% 3
       - %\at% 7 и 3
       4
     (* 3 2)
       3 %\as% 3
       2 %\as% 2
       * %\at% 3 и 2
       6
     equal %\at% 4 и 6
     Nil
\end{lstlisting}

\begin{lstlisting}[style={scheme}]
6. (equal (abs (- 2 4)) 3)
     (abs (- 2 4))
       (- 2 4)
         2 %\as% 2
         4 %\as% 4
         - %\at% 2 и 4
         -2
       abs %\at% -2
       2
     3 %\as% 3
     equal %\at% 2 и 3
     Nil
\end{lstlisting}

\clearpage
\section*{Задание 2}

\begin{lstinputlisting}[
	caption={Задание 2},
	label={lst:t4},
	style={lsp},
	linerange={12-13},
	]{../src/main.lsp}
\end{lstinputlisting}

Написать функцию, вычисляющую гипотенузу прямоугольного треугольника по заданным катетам 

\begin{lstlisting}[style={scheme}]
(calc-triangle-hyp 3 4)

  (sqrt (+ (* 3 3) (* 4 4)))
    (+ (* 3 3) (* 4 4))
      (* 3 3)
        3 %\as% 3
        3 %\as% 3
        * %\at% 3 и 3
        9
      (* 4 4)
        4 %\as% 4
        4 %\as% 4
        * %\at% 4 и 4
        16
      + %\at% 9 и 16
      25
    sqrt %\at% 25
    5
  
\end{lstlisting}

\clearpage
\section*{Задание 3}

Написать функцию, вычисляющую объём параллелепипеда по 3-м его сторонам и составить диаграмму её вычисления.

\begin{lstinputlisting}[
	caption={Задание 3},
	label={lst:t4},
	style={lsp},
	linerange={15-16},
	]{../src/main.lsp}
\end{lstinputlisting}

\begin{lstlisting}[style={scheme}]
(calc-par-vol 1 2 3)
	
  (* 1 2 3)
    1 %\as% 1
    2 %\as% 2
    3 %\as% 3
    * %\at% 1, 2, 3
    6
\end{lstlisting}

\section*{Задание 4}

Каковы результаты вычисления следующих выражений?

\begin{lstinputlisting}[
	caption={Задание 4},
	label={lst:t4},
	style={lsp},
	linerange={2-9},
	]{../src/main.lsp}
\end{lstinputlisting}

\clearpage
\section*{Задание 5}

Написать функцию {\texttt{longer-than}} от двух списков-аргументов, которая возвращает {\texttt{T}}, если первый аргумент имеет большую длину.

\begin{lstinputlisting}[
	caption={Задание 5},
	label={lst:t5},
	style={lsp},
	linerange={18-19},
	]{../src/main.lsp}
\end{lstinputlisting}

\section*{Задание 6}

Каковы результаты вычисления следующих выражений?

\begin{lstinputlisting}[
	caption={Задание 6},
	label={lst:t6},
	style={lsp},
	linerange={22-28},
	]{../src/main.lsp}
\end{lstinputlisting}

\section*{Задание 7}

Дана функция 

\begin{lstinputlisting}[
	caption={mystery},
	label={lst:func},
	style={lsp},
	linerange={31-32},
	]{../src/main.lsp}
\end{lstinputlisting}

Каковы результаты вычисления следующих выражений?

\begin{lstinputlisting}[
	caption={Задание 7},
	label={lst:t7},
	style={lsp},
	linerange={35-38},
	]{../src/main.lsp}
\end{lstinputlisting}

\section*{Задание 8}

Написать функцию, которая переводит температуру в системе Фаренгейта в температуру по Цельсию. $f = \cfrac{9}{5}\cdot c + 32$

\begin{lstinputlisting}[
	caption={Задание 8},
	label={lst:t8},
	style={lsp},
	linerange={41-42},
	]{../src/main.lsp}
\end{lstinputlisting}

\noindentКак бы назывался роман Р. Бредбери <<+451 по Фаренгейту>> в системе по Цельсию? -- <<232.78 по Цельсию>>

\section*{Задание 9}

Что получится при вычислении каждого из выражений?

\begin{lstinputlisting}[
	caption={Задание 9},
	label={lst:t9},
	style={lsp},
	linerange={45-51},
	]{../src/main.lsp}
\end{lstinputlisting}

\section*{Дополнительное задание 1}

Написать функцию, вычисляющую катет по заданной гипотенузе и другому катету прямоугольного треугольника и составить диаграмму её вычисления. 

\begin{lstinputlisting}[
	caption={Дополнительное задание 1},
	label={lst:e1},
	style={lsp},
	linerange={54-55},
	]{../src/main.lsp}
\end{lstinputlisting}

\begin{lstlisting}[style={scheme}]
(calc-cathet 5 4)
  (sqrt (- (* 5 5) (* 4 4)))
    (- (* 5 5) (* 4 4))
      (* 5 5)
	    5 %\as% 5
	    5 %\as% 5
        * %\at% 5 и 5
        25
      (* 4 4)
        4 %\as% 4
        4 %\as% 4
        * %\at% 4 и 4
        16
      - %\at% 25 и 16
      9
    sqrt %\at% 9
    3
\end{lstlisting}

\section*{Контрольные вопросы}

\subsection*{1. Базис языка {\texttt{Lisp}}}

Базис языка представлен:
\begin{itemize}
	\item структурами, атомами;
	\item функциями:\\
	{\texttt{atom, eq, cons, car, cdr,}}\\
	{\texttt{cond, quote, lambda, eval, label}}.
\end{itemize}

\subsection*{2. Классификация функций языка {\texttt{Lisp}}}

Функции в языке {\texttt{Lisp}}:
\begin{itemize}
	\item чистые (с фиксированным количеством аргументов) -- математические функции;
	\item рекурсивные функции;
	\item специальные функции -- формы (принимают произвольное количество аргументов или по разному обрабатывают аргументы);
	\item псевдофункции (создающие <<эффект>> - отображающие на экране процесс обработки данных и т. п.);
	\item функции с вариативными значениями, выбирающие одно значение;
	\item функции высших порядков -- функционалы (используются для построения синтаксически управляемых программ).
\end{itemize}

\subsection*{3. Способы создания функций}

С помощью макро определения \texttt{defun} или с использованием Лямбда-нотации (функция без имени).

\subsection*{4. Функции {\texttt{car, cdr}}}

Являются базовыми функциями доступа к данным. {\texttt{car}} принимает точечную пару или список в качестве аргумента и возвращает первый элемент или {\texttt{Nil}}, {\texttt{cdr}} -- возвращает все элементы, кроме первого или {\texttt{Nil}}.

\subsection*{5. Функции {\texttt{list, cons}}}

Являются функциями создания списков ({\texttt{cons}} -- базовая, {\texttt{list}} -- нет). {\texttt{cons}} создаёт списочную ячейку и устанавливает два указателя на аргументы. {\texttt{list}} принимает переменное число аргументов и возвращает список, элементами которого являются аргументы, переданные в функцию.