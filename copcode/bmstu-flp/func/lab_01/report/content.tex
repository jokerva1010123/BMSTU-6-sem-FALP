\chapter{Задания}

\section{Представить следующие списки в виде списочных ячеек (№1)}

Решение приложено к отчету на отдельном листе.

\clearpage

\chapter{Ответы на вопросы к лабораторной работе}

\section{Перечислите элементы языка}

Элементы языка --- атомы и точечные пары (структуры, которые строятся с помощью унифицированных структур - блоков памяти - бинарных узлов). Атомы бывают:
\begin{itemize}
    \item \textbf{символы} (идентификаторы) --- синтаксически представляют собой набор литер (последовательность букв и цифр, начинающаяся с буквы; могут быть связанные и несвязанные);
    \item \textbf{специальные символы} --- используются для обозначения <<логических>> констант (\texttt{T}, \texttt{Nil});
    \item \textbf{самоопределимые атомы} --- числа, строки - последовательность символов в кавычках (\texttt{"abc"}).
\end{itemize}

\section{Синтаксис элементов языка}

\texttt{Точечная пара ::= (<атом> . <атом>) | (<точечная пара> . <атом>) | (<атом> . <точечная пара>) | (<точечная пара> . <точечная пара>)}

\texttt{Список ::= <пустой список> | <непустой список>}, где 

\texttt{<пустой список> ::= () | Nil},

\texttt{<непустой список> ::= (<S-выражение>. <список>)},

Список --- частный случай S-выражения.

Синтаксически любая структура (точечная пара или список) заключается в круглые скобки:
\texttt{(A . B)} --- точечная пара.
\texttt{(A)} --- список из одного элемента.
Непустой список --- \texttt{(A . (B . (C . (D . Nil))))} или \texttt{(A B C D)}
Пустой список --- \texttt{Nil} или \texttt{()}.

Элементы списка могут быть списками, например --- \texttt((A (B C) (D (E)))). Таким образом, синтаксически наличие скобок является признаком структуры --- списка или точечной пары.

Любая непустая структура Lisp в памяти представляется  списковой ячейкой, хранящий два указателя: на голову (первый элемент) и хвост (все остальное).

\section{Как воспринимается <<'>>?}

Как спецальная функция \texttt{quote}. Данная функция блокирует вычисления своего единственного аргумента, то есть он воспринимается как константа. При выполнении функции аргумент обрабатывается по общей схеме.

\section{Что такое рекурсия (и где используется)?}

Рекурсия --- ссылка на описываемый объект в процессе его описания.

Рекурсия используется при работе со списками: каждый непустой список представлен точечной парой, состоящей из головы (которая может быть любым S-выражением) и хвоста (который является списком).

Так же может использоваться при работе с рекурсивными функциями, например, рекурсивная функция для получения самого левого элемента древовидной структуры: \texttt{(defun f (x) (cond ((atom x) x) (T (f (car x)))))}.
